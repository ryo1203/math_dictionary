\documentclass[dvipdfmx]{jsarticle}
\usepackage{mymacros}
\usepackage{amsmath,amssymb,amsthm}
\usepackage{tcolorbox}
\usepackage{mathrsfs}
\usepackage{tikz}
\usepackage{multicol}
\newtheorem{theo}{定理}
\newtheorem{defi}{定義}
\newtheorem{lemm}{補題}
\newtheorem{plob}{演習問題}
\newtheorem{axio}{公理}
\title{圏論辞書}
\author{石塚 伶}
\date{}
\begin{document}
\maketitle

\textbf{a}

  \begin{tabular}{ll}
    arrows & 射 \\
    asosiativity & 結合法則 \\
    Ab & アーベル群による圏 \\
    adjoint & 随伴 \\
    arbitrary & 任意の \\
  \end{tabular}

\textbf{b}

  \begin{tabular}{ll}
  \end{tabular}

\textbf{c}

  \begin{tabular}{ll}
    composition & 合成 \\
    commutative diagrams & 可換図式 \\
    commutes & 可換 \\
    codomain & 終域($\mathrm{cod}$) \\
    continuous maps & 連続写像 \\
    contravariant functor & 反変関手 \\
    covariant functor & 共変関手 \\
    CAT & 圏と関手による圏 \\
    CRing & 可換環による圏 \\
    collection & 系 \\
  \end{tabular}

\textbf{d}

  \begin{tabular}{ll}
    domain & 始域($\mathrm{dom}$) \\
    discrete & 離散 \\
    discrete category & 離散圏 \\
    dual category & 相対圏 \\
    dual & 相対 \\
  \end{tabular}

\textbf{e}

  \begin{tabular}{ll}
    equivalence(s) & 同値 \\
    equivalent & 対等($\simeq$) \\
    empty category & 空な圏(対象が$\emptyset$である圏) \\
    epimorphism & 全射 \\
    essentially surjective on objects & 対象における本質的全射 \\
  \end{tabular}

\textbf{f}

  \begin{tabular}{ll}
    full subcategory & 充満部分圏($\mathscr{C}'(A,B) = \mathscr{C}(A,B)$) \\
    functor & 関手 \\
    forgetful functor & 忘却関手 \\
    free functor & 自由関手 \\
    faithful & 忠実 \\
    full & 充満 \\
    functor category & 関手圏($[\mathscr{A},\mathscr{B}]$ \, , \, $\mathscr{B^{\mathscr{A}}}$) \\
    family & 族 \\
    finitely & 有限 \\
  \end{tabular}

\textbf{g}

  \begin{tabular}{ll}
    Grp & 群と群準同型写像による圏 \\
    group homomorphism & 群準同型写像 \\
  \end{tabular}

\textbf{h}

  \begin{tabular}{ll}
    homomorphism & 準同型写像 \\
    homomorphic & 準同型 \\
    horizontal composition & 水平合成 \\
  \end{tabular}

\textbf{i}

  \begin{tabular}{ll}
    identity & 恒等射($\mathrm{id},1$) \\
    identity laws & 単位法則 \\
    isomorphism & 同型射 \\
    isomorphic & 同型($A \cong B$) \\
    inverse & 逆射($f^{-1}$) \\
    identity functor & 恒等関手 \\
    injection & 集合の単射 \\
    monomorphism & 単射 \\
    injective & 単射 \\
    interchange law & (水平合成と垂直合成の)交換法則 \\
  \end{tabular}

\textbf{j}

  \begin{tabular}{ll}
  \end{tabular}

\textbf{k}

  \begin{tabular}{ll}
  \end{tabular}

\textbf{l}

  \begin{tabular}{ll}
    left adjoint & 左随伴 \\
  \end{tabular}

\textbf{m}

  \begin{tabular}{ll}
    maps & 射 \\
    morphisms & 射 \\
    monoid & モノイド \\
    metacategory & メタ圏 \\
    Mon & モノイドによる圏 \\
  \end{tabular}

\textbf{n}

  \begin{tabular}{ll}
    natural transformation & 自然変換 \\
    natural isomorphism & 自然同型(射) \\
    naturally isomorphic & 自然同型 \\
    naturally in A($\in \mathscr{A}$) & Aについて自然である
  \end{tabular}

\textbf{o}

  \begin{tabular}{ll}
    objects & 対象($\mathrm{ob}(\mathscr{A})$) \\
    opposite category & 相対圏 \\
    one-to-one map & 単射、一対一の写像 \\
    onto map & 上への写像 \\
  \end{tabular}

\textbf{p}

  \begin{tabular}{ll}
    preorder & 前順序 \\
    preordered set & 前順序集合 \\
    principle of duality & 双対原理 \\
  \end{tabular}

\textbf{q}

  \begin{tabular}{ll}
    quiver & クイバー(箙)
  \end{tabular}

\textbf{r}

  \begin{tabular}{ll}
    Ring & 環と環準同型写像による圏 \\
    Ring homomorphism & 環準同型写像 \\
    right adjoint & 右随伴 \\
  \end{tabular}

\textbf{s}

  \begin{tabular}{ll}
    Set & 集合と写像による圏 \\
    subcategory & 部分圏 \\
    surgection & 集合の全射 \\
    surjective & 全射 \\
  \end{tabular}

\textbf{t}

  \begin{tabular}{ll}
    Top & 位相空間と連続写像による圏 \\
  \end{tabular}

\textbf{u}

  \begin{tabular}{ll}
    union & 合併(和集合) \\
  \end{tabular}

\textbf{v}

  \begin{tabular}{ll}
    $\mathrm{Vect}_k$ & 体k上の線形空間と線型写像による圏 \\
    vertical composition & 垂直合成 \\
  \end{tabular}

\textbf{w}

  \begin{tabular}{ll}
    words & 語 \\
  \end{tabular}

\textbf{x}

  \begin{tabular}{ll}
  \end{tabular}

\textbf{y}

  \begin{tabular}{ll}
  \end{tabular}

\textbf{z}

  \begin{tabular}{ll}
  \end{tabular}

\end{document}
